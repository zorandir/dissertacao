% RESUMO
\newpage
\thispagestyle{plain}
%\pagenumbering{roman}
\begin{center}
\large
\textbf{RESUMO}
\end{center}
%\renewcommand{\baselinestretch}{0.6666666}
Templates são representações formais para conjuntos de autômatos celulares (ACs) feitas por meio da generalização das tabelas de transição clássicas. Já existem algoritmos que geram templates para propriedades estáticas %TODO: definir propriedades estáticas de maneira simples
e algoritmos que realizam operações como intersecção entre templates e expansão de template. Este projeto de pesquisa introduz a operação de complemento de templates e o funcionamento do algoritmo dessa operação implementada na biblioteca \textit{CATemplates}. Também são apresentados exemplos das possibilidades de uso do templates no problema de paridade, com o apoio das operações de complemento e das operações geradoras de templates de ACs conservativos de paridade e ACs conservativos de estado.
\\[0.5cm]
\begin{flushleft}
{\bf Palavras-chave:} {\it Autômatos celulares, templates, conservabilidade, problema de paridade.}
\end{flushleft}

% ABSTRACT
\newpage
\thispagestyle{plain}
%\pagenumbering{roman}
\begin{center}
\large  
\textbf{ABSTRACT}
\end{center}
%\renewcommand{\baselinestretch}{0.6666666}
Templates are formal representation for cellular automata sets (CAs) made through the generalization of the classical transitions table. There are algorithms that generate templates for static properties and algorithms that perform operations such as intersection between templates and template expansion. In this research project it is introduced the templates complement operating and explained the functioning of the algorithm of this operation implemented in the \textit{CATemplates} package. Also show examples of possibilities of using templates in the parity problem, with the support of complement operations and generating operations of CAs conservative parity and ACs state conservative.
\\[0.5cm]
\begin{flushleft}
{\bf Keywords:} {\it Cellular automata, templates, conservation, parity problem.}
\end{flushleft}