%!TEX root = qualificacao.tex
\section{RESULTADOS PARCIAIS}
\label{sec:resultadosParciais}

O presente projeto de pesquisa já apresenta como resultado o desenvolvimento da operação de complemento para templates com $k=2$, bem como a demonstração de uma série de passos que, utilizando a operação de complemento, conseguem restringir o espaço busca de ACs de raio 3 que tem a possibilidade de solucionar o problema de paridade.

A operação de complemento é responsável por obter um conjunto de templates que represente todas as regras não representadas por um template passado como argumento. Está operação pode ser melhor visualizada abaixo:
\begin{equation}
C(T_1)=\overline{T}_1
\end{equation}

A Figura \ref{fig:complement} ilustra essa operação que consiste em passar um template $T_1$ para a função, e receber um conjunto de templates complementares a $T_1$, aqui representados como $\overline{T}_1$.
\begin{figure}[h!]
  \centering
  \def\svgscale{0.5}
  \import{./img/}{fig_complement.pdf_tex}
  \caption{Em branco, $T_1$ representa uma família de ACs. Em cinza, $\overline{T}_1$ representa o complemento dessa família.}
  \label{fig:complement}
\end{figure}

O processo que o algoritmo usa para encontrar o conjunto complementar de um template é efetuado através de uma sequência de etapas. A primeira etapa consiste em igualar o template recebido com o template base do mesmo espaço, obtendo assim combinações lógicas de equações. Então o algoritmo remove as equações tautológicas e aplica uma operação de negação nas equações, no caso binário a operação de negação consiste apenas em efetuar as permutações $\rho = (0 \rightarrow 1, 1 \rightarrow 0)$ ao resultado final das equações. Neste momento o algoritmo troca o operador lógico $\wedge$ por $\vee$ e por fim esse sistema é então passado como argumento para a função Solve. O resultado da função Solve é um conjuntos com diversos conjuntos de substituições que são aplicados ao template base. Para melhor visualizar essas etapas, considere o template $T_1 = (x_7, x_6, x_5, 1 - x_1, x_3, x_2, x_1, 0)$ de $k=2$ e $r=1$. Esse template será igualado com o template base gerando um sistema de equações representado pela Eq. \eqref{eq:complement}.
\begin{equation}
\left\{\begin{matrix}
x_7 & = & x_7	\\ 
x_6 & = & x_6	\\ 
x_5 & = & x_5	\\ 
x_4 & = & 1 - x_1 \\ 
x_3 & = & x_3	\\ 
x_2 & = & x_2	\\ 
x_1 & = & x_1	\\ 
x_0 & = & 0
\end{matrix}\right.
\label{eq:complement}
\end{equation}

Esse sistema deve ser representado através de combinações lógicas de equações. A Eq. \eqref{eq:logicalComplement} é equivalente a Eq. \eqref{eq:complement} e também pode ser resolvido pela função Solve do \textit{Wolfram Language}.
\begin{equation}
\begin{split}
x_7 = x_7	\wedge  
x_6 = x_6	\wedge  
x_5 = x_5	\wedge  
x_4 = 1 - x_1 \wedge  \\
x_3 = x_3	\wedge  
x_2 = x_2	\wedge  
x_1 = x_1	\wedge  
x_0 = 0
\end{split}
\label{eq:logicalComplement}
\end{equation}

Antes de passar a Eq. \eqref{eq:logicalComplement} para a função Solve, o algoritmo elimina todas as equações tautológicas e troca todo operadores lógico $\wedge$ por $\vee$. Ao se executar essas etapas na Eq. \eqref{eq:logicalComplement}, obtêm-se a Eq. \eqref{eq:logicalComplement1} ilustrada abaixo:
\begin{equation}
x_4 = 1 - x_1 \vee x_0 = 0
\label{eq:logicalComplement1}
\end{equation}

Por fim é aplicado a operação de negação nas equações. No caso binário basta efetuar a permutação $\rho $, que também pode ser feita por meio da função $f(x) = 1 - (x)$. A Eq. \eqref{eq:logicalComplement2} representa a combinação lógica de equações resultante dessas operações.
\begin{equation}
x_4 = 1 - (1 - x_1) \vee  
x_0 = 1 - 0
\label{eq:logicalComplement2}
\end{equation}

Após a efetuado esses passos, a combinação lógica de equações resultante é passada como argumento para a função Solve que retorna o conjunto solução $S$. Nesse exemplo $S = \{\{x_4 \to x_1\},\{x_0\to 1\}\}$. Perceba que $S$ apresenta mais de um conjunto de substituições, por consequência ambos devem ser utilizados para realizar as substituições no template base. Essas substituições faz com que se obtenha o conjunto de templates $\{(x_7,x_6,x_5,x_1,x_3,x_2,x_1,x_0),(x_7,x_6,x_5,x_4,x_3,x_2,x_1,1)\}$. A união desses dois templates representa todas as regras não representadas pelo template $T_1$.

No exemplo dado com o template $T_1$ apenas as etapas da operação de complemento descritas anteriormente são necessárias, mas nem sempre é assim. 
A operação de complemento precisa também verificar se o template possui combinações de substituições que o levem a gerar regras inválidas. 
Caso exista, gera-se os \textit{templates de exceção} do template passado como argumento e o adiciona ao conjunto de templates complementares, sendo que \textit{templates de exceção} são os templates apresentam um conjunto de substituições que levem o template original a apresentar substituições fora do intervalo inteiro $[0, k-1]$.

Exemplificando, considere o template $T_2 = (x_7, x_6, x_5, 1 - x_1 - x_2, x_3, x_2, x_1, 0)$ para $k=2$.
A primeira etapa da operação de complemento ocorre normalmente e encontra os templates complementares $\{(x_7, x_6, x_5, x_4, x_3, x_2, x_1, 1),(x_7, x_6, x_5, x_1 + x_2, x_3, x_2, x_1, x_0)\}$.
Todavia é trivial perceber que qualquer expansão do template $T_2$ que tenha o conjunto de substituições $\{x_1 = 1, x_2 = 1\}$ fará com que a quinta posição do template apresente o valor $2$, que não pertence ao intervalo $[0,k-1]$.
Logo, todos os templates que apresentem $\{x_1 = 1, x_2 = 1\}$ são complementares ao template $T_2$.
Portanto, no caso de $T_2$, será obtido o template de exceção $T_{excecao 1} = (x_7, x_6, x_5, x_4, x_3, 1, 1, x_0)$. 
A operação de complemento então adiciona na lista de templates complementares os templates de exceção encontrados, finalizando assim o processo e obtendo o seguinte conjunto de templates complementares:
\begin{equation}
\{(x_7, x_6, x_5, x_4, x_3, x_2, x_1, 1),(x_7, x_6, x_5, x_1 + x_2, x_3, x_2, x_1, x_0),(x_7, x_6, x_5, x_4, x_3, 1, 1, x_0)\}
\label{eq:complementionSet}
\end{equation}

É importante dizer que a implementação do algoritmo que executa a operação de complemento ainda não permite trabalhar com $k\neq 2$, pois a negação das equações são feitas por meio da função $f(x) = 1 - (x)$. Outra questão relevante, que deve ser devidamente enfatizada, é que apesar de os exemplos com $T_1$ e $T_2$ não apresentarem mais de um templates de exceção, isso é possível e comum.

O desenvolvimento da operação de complemento de template permite diversas possibilidades de aplicação. 
Uma possibilidade interessante é no problema de paridade. 
Ainda não se sabe se existem regras de raio 3 que solucionem o problema de paridade \cite{Betel2013}. 
Mas os templates podem ser uma forma interessante de restringir o conjunto de regras na busca dessa solução.

As regras dos ACs que solucionam o problema de paridade têm algumas propriedades estáticas que podem ser trivialmente percebidas. 
Um AC que resolva o problema de paridade sempre será contido, tendo em vista no problema de paridade as vizinhanças homogêneas não devem levar a transições de estado ativas, e está é a única restrição de variável dos templates contidos para AC binários. 
Vale frisar que o espaço das regras contidas de raio 3 ainda é um espaço muito grande, entretanto essa não é a única propriedade estática que um AC que resolva o problema de paridade contém. 
Para que um AC resolva problema de paridade também é necessário que ele não seja conservativo, visto que se a soma dos estados do AC não mudar, ele nunca convergirá como propõe o problema. 
Por fim, espera-se que um AC que resolva o problema de paridade seja conservativo de paridade.

Dado a possibilidade de se obter os templates para todas essas propriedades, é também possível utilizar as operações de complemento e intersecção para restringir o espaço de busca para a solução desse problema.
Para efetuar esta restrição, basta efetuar a intersecção do templates de confinamento com o template de conservabilidade de paridade, visto que ambas as propriedades são consideradas interessante. 
Posteriormente, deve-se efetuar a operação de complemento do template das regras conservativas de estado, visto que essas regras não podem resolver o problema de paridade. 
E por fim, efetuar intersecção do template obtido pela primeira intersecção com cada um dos templates obtidos pela operação de complemento.

Formalmente essas operações de conjuntos entre os templates pode ser representado através da Eq. \eqref{eq:operationsTemplateParidade}, sendo que $T_{confinado}$ representa o templates das regras confinadas, $T_{conservaparidade}$ representa o templates das regras que conservam a paridade e $\overline{T}_{conservaestados}$ representa o conjunto de templates complementares ao templates das regras conservativas. O resultado dessa operação é $T_{paridade}$, esse resultado representa um conjunto de templates que restringem um pouco mais as regras com possibilidades de solucionar o problema de paridade.
\begin{equation}
T_{paridade} = (T_{conservaparidade} \cap T_{confinado}) \cap \overline{T}_{conservaestados}
\label{eq:operationsTemplateParidade}
\end{equation}