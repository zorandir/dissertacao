%!TEX root = dissertacao.tex
% RESUMO
\newpage
\thispagestyle{plain}
%\pagenumbering{roman}
\begin{center}
\large
\textbf{RESUMO}
\end{center}
%\renewcommand{\baselinestretch}{0.6666666}
\textit{Templates} são representações formais para conjuntos de autômatos celulares unidimensionais feitas por meio da generalização das tabelas de transição clássicas. Já existem algoritmos na literatura que geram templates para propriedades estáticas de autômatos celulares, assim como há algoritmos que realizam operações como intersecção entre templates e expansão de template. Aqui, é introduzida a operação de templates de exceção, a operação de diferença entre templates, e explica-se o funcionamento do algoritmo dessas operações, que foram implementadas na biblioteca \texttt{CATemplates} do software \textit{Mathematica}. Também discutimos a possibilidade de uso de templates no contexto do problema de paridade (a saber, a determinação da paridade de 1s em uma configuração binária cíclica de tamanho ímpar), com o apoio da operação de diferença entre templates, e das operações geradoras de templates de autômatos celulares conservativos de paridade e conservativos de estado.
\\[0.5cm]																																																																																																																																																																																																																																																			
\begin{flushleft}
{\bf Palavras-chave:} {\it Autômatos celulares, templates, diferença entre templates, templates de exceção, propriedades estáticas.}
\end{flushleft}
\pagenumbering{roman}


% ABSTRACT
\newpage
\thispagestyle{plain}
%\pagenumbering{roman}
\begin{center}
\large  
\textbf{ABSTRACT}
\end{center}
%\renewcommand{\baselinestretch}{0.6666666}
Templates are formal representations for sets of one-dimensional cellular automata created by means of a generalisation of the classical state transition tables. Algorithms already do exist in the literature that generate templates for static properties of cellular automata rules, as well as others that perform operations such as intersection between templates and template expansion. Here, we introduce the exception template operation, the operation of difference between templates, and explain the functioning of the algorithm of those operations, which have been implemented in the \texttt{CATemplates} package of the \textit{Mathematica} software. We also discuss the possibility of using templates in the context of the parity problem --namely, the determination of the parity of 1s in a cyclic binary configuration of odd length-- with the support of the operation of difference between templates, and of the template generation of parity conserving and state conserving cellular automata.
\\[0.5cm]
\begin{flushleft}
{\bf Keywords:} {\it Cellular automata, templates, difference between templates, exception template, static properties.}
\end{flushleft}